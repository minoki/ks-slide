\documentclass{beamer}
\usepackage{luatexja}
\usepackage{luatexja-ruby}
\usepackage{verbatim}
\usepackage[T1]{fontenc}
\usepackage{alltt}
\usepackage{xcolor}
\hypersetup{unicode}
\usepackage{fancyvrb}
\DefineVerbatimEnvironment{MyCode}{Verbatim}{numbers=left,numbersep=5pt,frame=leftline,showspaces,fontshape=up}
\DefineVerbatimEnvironment{MyCodeAllTT}{Verbatim}{numbers=left,numbersep=5pt,frame=leftline,fontshape=up,commandchars=\\\{\}}
\title{\TeX の字句解析}
\author{荒田 実樹}
\date{2016年6月9日}
\begin{document}
\begin{frame}\frametitle{}
  \titlepage
\end{frame}

\begin{frame}\frametitle{目次}
  \tableofcontents
\end{frame}

\section{\TeX の処理の概要}
\begin{frame}[fragile]\frametitle{\TeX の処理の概要}
  \TeX の処理は大きく4段階に分けられる。
  \begin{enumerate}
  \item 字句解析%(input processor)
    \begin{itemize}
    \item 入力ファイルから「トークン」を切り出す
    \item コメントを読み飛ばしたりもする
    \end{itemize}
  \item 展開%(expansion processor)
    \begin{itemize}
    \item マクロの展開
    \item 条件分岐
    \item なかなか奥が深い
    \end{itemize}
  \item 実行%(execution processor)
    \begin{itemize}
    \item 新しくマクロを定義する \\
      (plain \TeX なら \verb|\def|, \LaTeX なら \verb|\newcommand|)
    \item 字句解析器に介入できる(!)
    \item ほか、色々
    \end{itemize}
  \item 組版%(visual processor)
  \end{enumerate}
\end{frame}

\section{字句解析とは}

%\newcommand\ignoredspace{{\color{blue}\fboxsep=0.1pt\fboxrule=0.02pt\fbox{\vphantom{(\{M;}\phantom{ }}}}
\newcommand\ignoredspace{{\color{blue}\fboxsep=0.1pt\fboxrule=0.3pt\fbox{\vphantom{(\{M;}\color{black}\textvisiblespace}}}

\begin{frame}[fragile]\frametitle{字句解析とは?}
  普通のプログラミング言語のソースコードは、「トークン」と呼ばれる最小単位に分割できる。
\begin{overprint}\onslide<1>
\newcommand\tok[1]{{\fboxsep=0.7pt\fboxrule=0pt\fbox{\vphantom{(\{Mg;}#1}}}
\begin{quote}
\begin{MyCodeAllTT}
\tok{if} \tok{(}\tok{Math}\tok{.}\tok{PI} \tok{>} \tok{3.14}\tok{)} \tok{\{}
    \tok{console}\tok{.}\tok{log}\tok{(}\tok{"Hello world!"}\tok{)}\tok{;}
    // This is a comment
\tok{\}}
\end{MyCodeAllTT}
\end{quote}
\onslide<2>
  \newcommand\ctok[2]{{\color{#1}\fboxrule=0.7pt\fboxsep=0pt\fbox{\vphantom{(\{Mg;}\color{black}#2}}}
  \newcommand\kwd{\ctok{blue}}
  \newcommand\numlit{\ctok{red}}
  \newcommand\strlit{\ctok{brown}}
  \newcommand\punct{\ctok{lightgray}}
  \newcommand\ident{\ctok{gray}}
\begin{quote}
\begin{MyCodeAllTT}
\kwd{if} \punct{(}\ident{Math}\punct{.}\ident{PI} \punct{>} \numlit{3.14}\punct{)} \punct{\{}
    \ident{console}\punct{.}\ident{log}\punct{(}\strlit{"Hello world!"}\punct{)}\punct{;}
    // This is a comment
\punct{\}}
\end{MyCodeAllTT}
\end{quote}
トークンの種類: \\
\kwd{キーワード}、\punct{記号}、\ident{識別子}、\numlit{数値リテラル}、\strlit{文字列リテラル}、etc.

空白やコメントは読み飛ばされる。
\end{overprint}
\end{frame}

\begin{frame}[fragile]\frametitle{\TeX の場合は?}
  \TeX におけるトークンの種類:
  \begin{itemize}
  \item 普通の文字(例:\texttt{a}, \texttt{b}, \texttt{c}, \texttt{0}, \texttt{1}, \texttt{2}, \texttt{@}, \texttt{+}, \texttt{|})
  \item 空白文字
  \item 特殊文字 \verb|$ # ^ _ { } & ~ % \|
  \item コントロールシークエンス(例: \verb|\foo|)
    \begin{itemize}
    \item \TeX のコマンドの名前に使える
    \item 通常は、バックスラッシュ \texttt{\textbackslash} の後にアルファベットをいくつか続けたもの
    \item バックスラッシュの後に記号一文字というパターンもある(例:\verb|\!|, \verb|\\|)
    \end{itemize}
  \end{itemize}
  %\item 予約語とか、文字列リテラルとか、そういうのはない
  コメント(\verb|%| から始まる)および\alert{一部の空白}は、字句解析時に読み飛ばされる。
\end{frame}

\section{空白の扱い}

\begin{frame}\frametitle{空白の扱い}
  \TeX では、場所によって空白が無視されたり無視されなかったりする。
  例えば、以下の状況では空白が無視される。
  \begin{itemize}
  \item コントロールシークエンスの直後 \\
    (例:\texttt{\textbackslash foo\ignoredspace\{bar\}\textvisiblespace\{piyo\}})
  \item 行頭
  \item 空白文字の直後
  \item コマンドの引数の直前
    (例:\texttt{\textbackslash foo\textvisiblespace\{bar\}\ignoredspace\{piyo\}})
  \item 数式モード中の空白文字
    (例:\texttt{\$\ignoredspace a+b\ignoredspace\$})
  \end{itemize}
\end{frame}

\begin{frame}\frametitle{空白の扱い}
  どの段階で無視される?
  \begin{enumerate}
  \item 字句解析で無視される空白(このスライドで解説)
    \begin{itemize}
    \item コントロールシークエンスの直後
      (例:\texttt{\textbackslash foo\ignoredspace\{bar\}\textvisiblespace\{piyo\}})
    \item 行頭
    \item 空白文字の直後
    \end{itemize}
  \item 展開時に無視される空白
    \begin{itemize}
    \item コマンドの引数の直前など
      (例:\texttt{\textbackslash foo\textvisiblespace\{bar\}\ignoredspace\{piyo\}})
    \end{itemize}
  \item 実行時に無視される空白
    \begin{itemize}
    \item 数式モード中の空白文字など
      (例:\texttt{\$\ignoredspace a+b\ignoredspace\$})
    \end{itemize}
  \end{enumerate}
\end{frame}

\begin{frame}[fragile]\frametitle{空白の扱い --- 無視される状況 その1}
  コントロールシークエンスの直後の空白%(改行が空白文字とみなされる場合も含む)
は無視される。

  \begin{itemize}
  \item \verb|\TeX!!!| と書いても \texttt{\textbackslash TeX\ignoredspace!!!} と書いても同じ。
  \item ただし、記号一文字の場合は直後の空白は無視されない。
    \begin{itemize}
    \item \verb*|keisan\%sugaku| → \fbox{keisan\%sugaku}
    \item \texttt{keisan\textbackslash\%\textvisiblespace sugaku} → \fbox{keisan\% sugaku}
    \end{itemize}
  \item 空白文字一文字の場合は直後の空白は無視される。
    \begin{itemize}
    \item \verb*|\renewcommand{\ }{*}| した状況で、
    \item \verb*|keisan\ sugaku| → \fbox{{\renewcommand{\ }{*}keisan\ sugaku}}
    \item \texttt{keisan\textbackslash\textvisiblespace\ignoredspace sugaku} → \fbox{{\renewcommand{\ }{*}keisan\  sugaku}}
    \end{itemize}
  \end{itemize}
\end{frame}

\begin{frame}[fragile]\frametitle{空白の扱い --- 無視される状況 その2}
  行頭の空白は無視される。

  \begin{block}{例}
以下の2つは等価。
    \begin{quote}
\begin{MyCode}
Happy
\TeX{} Life!
\end{MyCode}
\end{quote}
    \begin{quote}
\begin{MyCodeAllTT}[showspaces]
Happy
\ignoredspace\ignoredspace\textbackslash{}TeX\{\} Life!
\end{MyCodeAllTT}
\end{quote}
  \end{block}
  ちなみに、この例では \verb|\TeX| の後の空白を無視させないために \verb|{}| を挟んでいる。
\end{frame}

\defverbatim\codeA{
\begin{verbatim*}
a
b
\end{verbatim*}
}

\section{改行の扱い}
\begin{frame}[fragile]\frametitle{改行の扱い}
改行は1個の半角空白として取り扱われる。
\begin{block}{例}
  以下の2つは等価。
  \begin{quote}
\begin{MyCode}
a
b
\end{MyCode}
\end{quote}
\begin{quote}
\begin{MyCode}
a b
\end{MyCode}
\end{quote}
\end{block}
  \begin{itemize}
\item 半角空白を入れずにソースコード上で改行したい場合は、
  \begin{quote}
\begin{MyCode}
a%
b
\end{MyCode}
\end{quote}
という風に、コメントを使うと良い。(\texttt{ab} と等価)
\item 「コントロールシークエンスの直後の空白は無視される」ので、コントロールシークエンスの直後の\alert{改行}も無視される。
  \end{itemize}
\end{frame}

\begin{frame}[fragile]\frametitle{改行の扱い --- p\TeX の場合}
p\TeX では、和文の直後の改行は無視される。
  \begin{itemize}
  \item さっきのルールによれば、以下の2つは\alert{\TeX 的には}同じはず。
    \begin{quote}
\begin{MyCode}
わぁい
うすしお
\end{MyCode}
\end{quote}
    \begin{quote}
\begin{MyCode}
わぁい うすしお
\end{MyCode}
\end{quote}
  \item しかし、\alert{p\TeX では}違う:前者は空白が入らない。
  \end{itemize}
\end{frame}

\begin{frame}[fragile]\frametitle{改行の扱い --- 改段落}
空行は、``\texttt{\textbackslash par}'' というコントロールシークエンスとして扱われる。
\begin{block}{例}以下の2つは等価。
\begin{quote}
\begin{MyCode}
「イヤーッ!」

「グワーッ!」
\end{MyCode}
\end{quote}
  \begin{quote}
\begin{MyCode}
「イヤーッ!」
\par
「グワーッ!」
\end{MyCode}
\end{quote}
\end{block}
\begin{itemize}
\item \texttt{\textbackslash par} は通常は改段落コマンドを表す。
\item 空行が2つあったら \texttt{\textbackslash par} 2つになる。
  (改段落コマンドは2つ以上あっても効果は同じ)
\end{itemize}
\end{frame}

\begin{frame}[fragile]\frametitle{改行の扱い --- 改段落}
空白のみからなる行も空行として扱われる(``\texttt{\textbackslash par}'' になる)。
\begin{block}{例}以下の2つは等価。つまり、どちらも改段落される。
\begin{quote}
\begin{MyCode}
「イヤーッ!」

「グワーッ!」
\end{MyCode}
\end{quote}
  \begin{quote}
\begin{MyCodeAllTT}
「イヤーッ!」
\textvisiblespace\textvisiblespace\textvisiblespace
「グワーッ!」
\end{MyCodeAllTT}
\end{quote}
しかし、次のように、コメントのみからなる行は改段落にならない。
  \begin{quote}
\begin{MyCode}
「イヤーッ!」
%
「グワーッ!」
\end{MyCode}
\end{quote}
\end{block}
\end{frame}

\section{字句解析への介入}
\begin{frame}\frametitle{字句解析への介入}
  \TeX では、実行中のプログラムが自身の字句解析に介入できる!

  \begin{itemize}
  \item \TeX のプログラムの実行結果によって、特殊文字の扱いを変更できる。
  \item あるいは、新たに特殊文字を作ったりできる。
  \end{itemize}

\end{frame}

\begin{frame}[fragile]\frametitle{字句解析への介入 --- 例}
  \LaTeX のコマンドにも、字句解析に介入するものがある。
  \begin{itemize}
  \item 特殊文字を無効化するコマンド・環境
    %(``\verb|\verb+\hello world+|'' と書いても、``\verb|\hello|'' はコントロールシークエンスとして扱われない)
    \begin{itemize}
    \item \verb|\verb| コマンド
    \item \texttt{verbatim} 環境
    \item \texttt{comment} 環境
    \item \texttt{alltt} 環境:\verb|\|, \verb|{|, \verb|}|以外の特殊文字を無効化する
      % →コマンドを記述できる
    \end{itemize}
  \item \verb|\makeatletter|, \verb|\makeatother| コマンド
    \begin{itemize}
    \item コマンド名に ``\texttt{@}'' を使えるようにする。
      %(例:\verb|\@ifnextchar|)
    \end{itemize}
  \end{itemize}
\end{frame}
\begin{frame}[fragile]\frametitle{字句解析への介入 --- 制限}
  すでに字句解析が終わった部分へは介入できない。

\newcommand\ctok[2]{{\color{#1}\fboxrule=0.3pt\fboxsep=0pt\fbox{\vphantom{(\{Mg;}\color{black}#2}}}
\newcommand\tokControlSequence[1]{\ctok{red}{\texttt{#1}}}
\newcommand\tokCS[1]{\ctok{red}{\texttt{\string#1}}}
\newcommand\tokOther[1]{\ctok{brown}{\texttt{#1}}}
\newcommand\tokLetter[1]{\ctok{gray}{\texttt{#1}}}
\newcommand\tokSp[1]{\ctok{yellow}{\texttt{#1}}}
\newcommand\tokSpace{\ctok{blue}{\textvisiblespace}}

\begin{block}{例}
\begin{quote}
%\newcommand{\foo}[1]{#1}
%\foo{\verb+\hello world+}
\begin{overprint}\onslide<1>
\renewcommand\ctok[2]{{\fboxrule=0.0pt\fboxsep=0.3pt\fbox{\vphantom{(\{Mg;}#2}}}
\begin{MyCodeAllTT}
\tokCS\newcommand\tokSp\{\tokCS\foo\tokSp\}\tokOther[\tokOther1\tokOther]\tokSp\{\tokSp#\tokOther1\tokSp\}
\tokCS\verb\tokOther+\tokOther\textbackslash\tokLetter{h}\tokLetter{e}\tokLetter{l}\tokLetter{l}\tokLetter{o}\tokOther{\textvisiblespace}\tokLetter{w}\tokLetter{o}\tokLetter{r}\tokLetter{l}\tokLetter{d}\tokOther+
\tokCS\foo\tokSp\{\tokCS\verb\tokOther+\tokCS\hello\textvisiblespace\tokLetter{w}\tokLetter{o}\tokLetter{r}\tokLetter{l}\tokLetter{d}\tokOther+\tokSp\}
\end{MyCodeAllTT}
\onslide<2>
\begin{MyCodeAllTT}
\tokCS\newcommand\tokSp\{\tokCS\foo\tokSp\}\tokOther[\tokOther1\tokOther]\tokSp\{\tokSp#\tokOther1\tokSp\}
\tokCS\verb\tokOther+\tokOther\textbackslash\tokLetter{h}\tokLetter{e}\tokLetter{l}\tokLetter{l}\tokLetter{o}\tokOther{\textvisiblespace}\tokLetter{w}\tokLetter{o}\tokLetter{r}\tokLetter{l}\tokLetter{d}\tokOther+
\tokCS\foo\tokSp\{\tokCS\verb\tokOther+\tokCS\hello\textvisiblespace\tokLetter{w}\tokLetter{o}\tokLetter{r}\tokLetter{l}\tokLetter{d}\tokOther+\tokSp\}
\end{MyCodeAllTT}
\end{overprint}
\end{quote}
\end{block}

\begin{itemize}
\item 2番目の \verb+\verb+ は期待通りに動作するか?
  \onslide<2> $\rightarrow$ No!
%\item \verb|\verb| が実行されるのは ``\verb|{\verb+\hello world+}|'' の字句解析が終わった後。
\item 2番目の \verb|\verb| が実行される時点で、後ろの部分が %
\tokOther{+}
\tokControlSequence{\textbackslash hello}
\tokLetter{w} \tokLetter{o} \tokLetter{r} \tokLetter{l} \tokLetter{d}
\tokOther{+}
という風に字句解析済み。
\item 特殊文字を無効化できずに、エラーになる。
\item \texttt{\textbackslash verb} 等の\alert{「字句解析に介入する」コマンドは、マクロの引数の中では使えない}(重要)
\end{itemize}
\end{frame}

\begin{frame}[fragile]\frametitle{カテゴリーコード}
  \TeX では、字句解析時に読み取った文字に対して「カテゴリーコード」と呼ばれる整数を割り当てる。 \\
  この「カテゴリーコード」によって、その文字がどう扱われるかが決まる。
  \begin{enumerate}
    \setcounter{enumi}{-1} % 0から始める
  \item Escape character: \verb|\|
  \item Beginning of group: \verb|{|
  \item End of group: \verb|}|
  \item Math shift: \verb|$|
  \item Alignment tab: \verb|&|
  \item End of line
  \item Parameter character: \verb|#|
  \item Superscript: \verb|^|
  \item Subscript: \verb|_|
    %\item Ignored
    \item[] (続く)
  \end{enumerate}
\end{frame}

\begin{frame}[fragile]\frametitle{カテゴリーコード}
  \begin{enumerate}
  \item[] (続き)
    \setcounter{enumi}{9}
  \item Space
  \item Letter: アルファベット。
    \begin{itemize}
    \item コントロールシークエンスを読む際に、アルファベットとして扱われる。
    \end{itemize}
  \item Other: 記号や数字。
  \item Active: 半角チルダ(\verb|~|).
    \begin{itemize}
    \item バックスラッシュなしで、単独でコマンド名として使える文字の事。
      \verb|\newcommand{~}{...}| みたいなことができる。
    \item これに対し、半角チルダ以外の記号、例えば @ では \verb|\newcommand{@}{...}| とはできない。
    \end{itemize}
  \item Comment character: \verb|%|
  %\item Invalid character
  \end{enumerate}
\end{frame}

\begin{frame}[fragile]\frametitle{字句解析への介入 --- 悪用例}
  カテゴリーコードを変更すると、例えば、普通の文字・記号をコマンド名として使える。

\begin{block}{例}
文章中の句読点「、。」を、カンマとピリオド「,.」に変える
\begin{quote}
\begin{MyCode}
\catcode`、=13 \def、{,}%
\catcode`。=13 \def。{.}%
最近、学校が好きだ。%→「最近,学校が好きだ.」になる
\end{MyCode}
\end{quote}
%この例だと、素直に日本語入力の設定を変えるか、エディターで置換するべき
\end{block}

%「自分が何をやっているかわかっていて」「思わぬ副作用にも対処できる」上級者向け
「自分が何をやっているかわかっていて」「何が起こっても責任を取れる」上級者向け。
\end{frame}

\begin{frame}[fragile]\frametitle{説明しなかったこと&参考文献}
  \begin{block}{字句解析について説明しなかったこと}
    \begin{itemize}
    \item カテゴリーコード 9 (ignored) とカテゴリーコード 15 (invalid character)
    \item Superscript 2つ(\verb|^^|)による制御文字の挿入(例:\verb|^^M|)
    \end{itemize}
  \end{block}

  \begin{block}{参考文献}
    \begin{thebibliography}{9}
    \item Donald Knuth, \emph{The \TeX book}, Addison-Wesley, 1984. \\
      特に Chapter 7: How \TeX{} Reads What You Type が字句解析について扱っている。
    \item Victor Eijkhout, \emph{\TeX{} by Topic}, Addison-Wesley, 1991. \\
      \href{http://eijkhout.net/texbytopic/texbytopic.html}{著者のWebサイト}からPDFを無料でダウンロード可能。
    \end{thebibliography}
  \end{block}
\end{frame}

\end{document}
